\section{Conclusion}

In order to predict Airbnb prices in Oslo we constructed our own Deep Neural Network and compared its performance to a collection of well-established Machine Learning algorithms.
Since we extracted numeric features out of the unstructured image and text data as part of the preprocessing pipeline, the input data fed into the models was of tabular nature.
Thus, classical Machine Learning models and in particular highly intepretable linear models indicated a competitive predictive performance on out-of-sample data.

Further, we emphasized the impact of using different feature set \emph{sizes} both on the training and validation set and used a two-dimensional feature embedding to visualize and explain the sensitivity of the network's predictions with respect to outliers in the price distribution.
The lowest \emph{test} set error was achieved by a straightforward ensemble of multiple individual predictive models.

There are various options to extend our work.
One particularly appealing idea for \emph{understanding} the Neural Net's behaviour is the construction of \emph{adversarial examples}.
In the regression context one could try to leverage steep areas in the high-dimensional loss surface such that small perturbations of each input feature result in exploding price predictions.

Bounding the magnitude of each perturbation by a constant might not be directly transferable to regression tasks in presence of categorical features.
As an alternative to the traditional approach borrowed from image classification which is often based on the \emph{Fast Gradient Sign Method} \citep{goodfellow2015}, a well-behaved Linear Regression model could again be used as a benchmark model.
Then, any popular gradient \emph{ascent} algorithm can be used to find feature combinations that maximize the distance between the Linear Regression prediction and the Neural Net prediction.
Since this approach does not naturally bound the input changes, the resulting loss function has to be heavily regularized to prevent pathological solutions.


