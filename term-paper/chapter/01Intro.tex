\section{Introduction}

% Describe the dataset and and define the problem. Do not explain in detail the basics of ANNs as we did in the introductory lecture.

% \item Einleitungssatz: Erster Satz, Führt zum Thema hin und weckt Interesse
% COMMENT: Start?

% \item Thema (2-3 Sätze nach Einleitungssatz) Relevanz, Motivation, Thema abgrenzen:
% COMMENT: Relevanz, Motivation und Thema abgrenzen??
The dataset provided by Airbnb includes various information about the offered accommodations in Oslo, Norway. It contains several individual records related to the details of the offer, the uploaded images and the assigned reviews of the accommodations.
%  \item Ziel (2-3 Sätze nach Thema) Was ist das Ziel der Arbeit?
The goal of this work now is to establish a deep learning approach to predict the price of an accommodation per night in this city. Thereby, especially the explainability and interpretability of the model is in focus. 

% \item Vorgehensweise / Methode (3-5 Sätze zusammen mit dem Ziel) Wie wird das Ziel der Arbeit erreicht? quantitativ / qualitativ?\\
First, to get all the relevant information from the data, we performed feature engineering to mainly extract additional features from the given images of the accommodations and the ratings. Then, we selected the relevant predictors for our model from these and the already given data using different feature selection algorithms. After preprocessing the data we set up an fully connected feed-forward neural network to predict the prices of the accommodations per night. In addition, to be able to evaluate the performance of our neural network, we implemented several classical machine learning models. We then compared the results of the different approaches and in a further step determined the importance of the individual variables. Finally, we considered the sensitivity of our neural network to outliers in the data with the help of an Variational Autoencoder, to create two-dimensional latent space representations, to illustrate to illustrate what difficulties the outliers cause.

% \item Aufbau (3-4 Sätze am Ende) Überblick und Aufzählung der Kapitel
Accordingly, this work is structured. In the first chapter, we will focus on the data and the associated preprocessing steps as well as give an overview of the methods used. Then, in the next section, we will have a detailed look on the predictive performance of the different models and their possible interpretability. Finally, we will discuss the results in the conclusion. 

% \item Forschungsstand (5-6 Sätze zusammen mit Thema) aktuelle Literatur




% \citep{Goodfellow-et-al-2016}

% \cite{hope2017, chollet2018}


% \begin{figure}[H]
%     \centering
%     %    \includegraphics{picture/xkcd.png}
%     \caption{Caption}
%     \label{fig:my_label}
% \end{figure}

























