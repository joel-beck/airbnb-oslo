\documentclass[12pt, letterpaper]{article}
\usepackage[utf8]{inputenc}
\usepackage{amsmath}
\usepackage{parskip}

\usepackage{graphicx}
\graphicspath{{../images/}}

\usepackage{hyperref}
\hypersetup{
    colorlinks=true,
    urlcolor=blue,
    linkcolor=blue,
    citecolor=blue,
    filecolor=blue,
}
\urlstyle{same}

\title{Paper Structure}
\author{}
\date{}


\begin{document}

\maketitle
\tableofcontents
\setcounter{tocdepth}{3}

\section{Introduction} % 0.5 Pages

% ---------------------------------------------------------------------------- %

\section{Methods} % Total of 9 + 5 = 14 Pages

% ---------------------------------------------------------------------------- %

\subsection{Preprocessing} % Total of 5 + 3 + 1 = 9 Pages

\subsubsection{Feature Engineering} % Total of 5 Pages

\textbf{Images} % 3 Pages (Marei sagt 2 plus Bild :) )

%%%
\begin{itemize}
    \item Discuss if figure of cnn examples can be moved to appendix
\end{itemize}
%%%

\textbf{Reviews} % 2 Pages
%%%
\begin{itemize}
    \item Description of Sentiment Analysis, stating procedure and results and including \textbf{Figure} with Wordcloud, either only English Words or Side-by-Side Wordclouds of English and Norwegian Words
    \item In addition: Language Detection to include the \emph{number of different languages} and the \emph{fraction of norwegian languages} and Analyzing the reviews lengths to include the \emph{median review length}
    \item Since there are multiple reviews per apartment the results for each review were averaged for each apartment separately.
\end{itemize}
%%%

\textbf{Others} % 0-1 pages
%%%
\begin{itemize}
    \item Optionally mention all other features that we added to the dataset
    \item All self-engineered features from images, from reviews and from existing metric variables were combined into a single dataframe as the foundation of all further analysis
\end{itemize}
%%%


\subsubsection{Feature Selection} % 3 Pages


\subsubsection{Price Distribution} % 1 Page

%%%
\begin{itemize}
    \item Discuss if figure of price distribution can be moved to appendix
\end{itemize}
%%%


% ---------------------------------------------------------------------------- %

\subsection{Models} % Total of 1 + 4 = 5 Pages

\subsubsection{Classical Models} % 1 Page
%%%
\begin{itemize}
    \item serve as benchmark models to better evaluate performance of custom neural network
    \item selected with increasing degrees of complexity and corresponding decreasing degree of interpretability
    \item Focus on $4$ models: \texttt{LinearRegression}, \texttt{Ridge}, \texttt{RandomForest} and \texttt{HistGradientBoosting}
    \item Describe Model Fitting process and hyperparameter tuning with Randomized Search Cross Validation
\end{itemize}
%%%

\subsubsection{Neural Network} % 4 Pages

%%%
\begin{itemize}
    \item Discuss if figure of dropout impact can be moved to appendix
\end{itemize}
%%%


% ---------------------------------------------------------------------------- %

\section{Results} % Total of 3 + 3 = 6 Pages

% ---------------------------------------------------------------------------- %

\subsection{Predictive Performance} % 3 Pages
%%%
\begin{itemize}
    \item \textbf{Figure} of performance comparison between selected classical models and neural network for given feature selector (e.g. \texttt{RFE}) and different number of selected features
    \item Interpret Differences in Training and Validation Performance between different models
    \item Interpret Differences in Performance for different number of selected features
    \item Compare Performance on Validation Set with Performance on Test Set for the best model of each class by means of a table \\
          $\Rightarrow$ Models whose hyperparameters were tuned on validation set generalize worse to test set, e.g. \texttt{HistGradientBoosting}, \texttt{RandomForest} and \texttt{Ridge}
    \item Include average predictions of top 2/3/4/5 models, where models are selected based on validation set performance and Test Set predictions are averaged
    \item Potentially mention which models contributed to predictions on new, unseen dataset from challenge (only in presentation)
\end{itemize}
%%%

% ---------------------------------------------------------------------------- %

\subsection{Explanations and Interpretation} % Total of 3 Pages

%%%
\begin{itemize}
    \item Discuss if coefficient plot can be moved to appendix
\end{itemize}
%%%

% ---------------------------------------------------------------------------- %

\section{Conclusion} % 0.5 Pages

% ---------------------------------------------------------------------------- %

% ask for order of appendix and references
\section{Appendix}
%%%
\begin{itemize}
    \item include link to repository with codebase to reproduce all findings
    \item include images of: % insert image names after discussion %
\end{itemize}
%%%

% ---------------------------------------------------------------------------- %

\section{References}

\end{document} % Total of 0.5 + 14 + 6 + 0.5 = 21 Pages

